\documentclass{article}
\usepackage[utf8]{inputenc}


\title{INFO1111:Computing 1A Professionalism}
\author{Semester 1 2021\thanks{}}

\date{Project 2B}
\begin{document}
	
	
	\begin{titlepage}
		\maketitle
		\begin{title}
			\begin{center}
				\textbf{Group Member:}
			\end{center}
		\end{title}
		\begin{center}
			
			\begin{tabular}{|c|c|c|}
				\hline &Full Name& Student ID\\
				\hline 1& Freya Liu & 500178278  \\
				\hline 2& Junbin Gou & 500685031  \\
				\hline 3&Full Name& Student ID  \\
				\hline 4& Louck Xu & 510311698  \\
				\hline 5&Full Name& Student ID  \\
				\hline
			\end{tabular}
		\end{center}
	\end{titlepage}
	
	\section{Introduction}
	We are group 5 in TUT16. Targeting to research the non-computing electives available at The University of Sydney, we choose some domains and electives to explore. Then we select the best non-computing electives suitable for these computing majors - Computer Science, Data Science, Information System, Software Development, Artificial Intelligence. \\
	\\
	In this report, we are going to introduce the optional course that we select and interpret why they are appropriate for the students in corresponding majors to learn.
	\section{Major Allocation}
	\begin{center}
		\begin{larger}
			\begin{tabular}{|c|c|c|}
				\hline &Full Name& Major\\
				\hline 1& Freya Liu & Computer science  \\
				\hline 2& Junbin Gou & Artificial Intelligence  \\
				\hline 3&Full Name& Student ID  \\
				\hline 4& Louck Xu & Software Development  \\
				\hline 5&Full Name& Student ID  \\
				\hline
			\end{tabular}
		\end{larger}
	\end{center}
	
	\section{Recommendations}
	\subsection{Computer Science}
	\subsection{Data Science}
	\subsection{Information Systems}
	\subsection{Software Development}
	\subsection{Artificial Intelligence}
	The two elective courses of Stochastic Process (STAT3021) and Optimisation and Financial Mathematics (MATH2070) belong to the field of financial mathematics and financial engineering. Stochastic Process specifically refers to a set of random variables. Stock fluctuations, voice signals, video signals, Brownian motion, etc. are all examples of stochastic processes in the real world. Optimisation and Financial Mathematics specifically refers to the application of probability theory and mathematical statistics, optimization methods, linear algebra and other related mathematical theories and methods, according to the established target return and risk tolerance, the process of recombining investments and diversifying risks. It reflects the process of investors’ willingness and the constraints imposed on investors, which are the maximization of returns under a certain level of risk or the minimization of risks under a certain level of return.\\
    \\
    Optimisation and Financial Mathematics will let us learn various portfolio models and optimization methods, such as linear programming, non-standard LP problems and two-phase simplex algorithm and nonlinear optimisation with and without constraints. At the same time, we can also learn about financial investment models, such as Utility theory and Capital asset pricing model, etc. \\
    \\
    Stochastic Process will let us learn all kinds of knowledge about random variables and various random process models and analysis methods, which are mainly divided into two parts: probability methods and mathematical analysis methods. The former includes orbital properties, stopping time, and stochastic differential equations, while the latter includes measurement theory, differential equations, semi-group theory, function stacks, and Hilbert space. The main focus of this course is on stationary processes, Markov processes, martingale theory, limit theorem and stochastic differential equations. Finally, we will learn some examples of the combination of stochastic process and real finance.\\
    \\
    These two courses can help us better combine the application of artificial intelligence and reality. For instance, when we deploy computing methods that support artificial intelligence, we need to model stochastic processes. There are many examples of stochastic processes in artificial intelligence, such as random walks and Brownian motion processes, both of which can be used in trading algorithms. Trading is another popular area of ​​artificial intelligence applications. Taking into account the growth rate and complexity of transactions, artificial intelligence technology is becoming an indispensable part of transaction practice. A particularly attractive feature of artificial intelligence is its ability to process large amounts of data to generate trading signals. Algorithms can be trained to automatically execute transactions based on these signals, which gave birth to the quantitative trading industry. In addition, artificial intelligence technology can reduce transaction costs by automatically analyzing the market and then determining the best transaction time, scale and method. Therefore, artificial intelligence also has a huge impact on portfolio management. Since the 2008 global financial crisis, the optimal management and risk control of asset portfolios have been at the forefront of asset management practices. As financial assets and global markets become more complex, traditional models may no longer be sufficient for asset allocation analysis. At the same time, artificial intelligence technology through the use of data learning and development can provide additional tools for optimizing asset portfolios. In conclusion, artificial intelligence can well assist investment managers in asset allocation and back-testing of different portfolio risk models.\\
    \\
    In the future professional career, if we choose to use artificial intelligence for quantitative investment, then the stochastic process and portfolio management and optimization will help us better integrate artificial intelligence and future professional career. Let's take the multi-factor model in quantitative investment as an example, where artificial intelligence can be used, one is to construct factors, and the other is to assign weights to factors. The construction factor here is to use machine learning methods to mine as much information as possible in the existing database. For example, based on the traditional EP and DP (the reciprocal of the price-to-earnings ratio and the price-to-book ratio), we explore whether there is valid information such as EP*DP, or EP raised to the DP power. In other words, it is to use artificial intelligence's fast and large-scale processing capabilities to mine useful information as much as possible. However, this kind of method seems reasonable, but it is inconsistent with the traditional investment logic. The traditional logic is that I know what exactly I want to invest in, and the computer is just a tool for me to implement. In other words, factors such as EP or DP have a strong theoretical basis behind their construction. The artificial intelligence searching for factors is searching for unknown islands in the ocean. Even if we find it, we don't know whether this is the result of data fitting or whether it is really effective. Furthermore, even if we gain good results, in real trading, it takes a lot of confidence to really use this factor. But when we combine the knowledge of stochastic process, use probabilistic methods and measurement theory to deduce, regard the price changes of the transaction target as a financial time series sequence, and decompose and reconstruct it, and make predictions of the sequence on this basis. This will make our model more valid and greatly improve our work efficiency and accuracy.\\
    \\
    When we have successfully developed an investment strategy, we always need to optimize the parameters of this strategy. At this time, we need to use the knowledge of the optimization strategy, and we can learn from Optimisation and Financial Mathematics. The basics of these knowledge are learned in the class. The convex optimization theory involved in this course, or more broadly, the optimization theory, is used in current artificial intelligence, data mining, or deep learning neural networks. The status of convex optimization theory is equivalent to the human backbone, and it supports the learning process of the entire model. Because models, in general, are like people learning to think. We know what we should learn, how we should learn, and how to adjust our knowledge when we find that we have learned it wrong, but no elements in computers is so smart, knowing what to learn and where learn. The optimization theory is a tool that tells the model what to learn and how to learn. The model learning is often a mapping function, that is, the parameter W in the model. The quality of this parameter depends on the answer. But after knowing what is wrong, where to learn, how to learn, and how to adjust, which is the role of optimization theory in it. If there is no optimization theory, then the model does not know how to learn, that is, without optimization, the learning of the model will always be stagnant.\\
    \\
    After we have studied these two elective courses, we will have more opportunities to choose employment, and at the same time, we will have more research topics to choose from. In our career path planning, we can include quantitative hedge funds. The investment position construction of quantitative hedge funds mainly relies on quantitative investment models. Compared with the traditional qualitative analysis-based combination construction idea, the quantitative model has obvious advantages in objectivity and rigor. It can cover a wider range of stocks, more repeatable strategy logic, and research and analysis process. The efficiency is also higher. The abilities of compound talents required by quantitative hedge funds mainly lie in three fields: computer, mathematics and finance. It happens that our Optimisation and Financial Mathematics corresponds to the financial field and part of the field of mathematics, and the Stochastic Process corresponds to the field of mathematics. Artificial intelligence, which is our major, corresponds to the computer field. When we have these three abilities and know and understand in all three areas, we can get the opportunity to work in these quantitative hedge funds, such as Bridgewater Associates, Citadel and so on.\\
    \\
    In terms of research, we will have more opportunities to combine artificial intelligence with financial and stochastic processes to construct better investment models based on time series, such as the 2017 article "Forecasting Foreign Exchange Rate Movements with k-Nearest- Neighbour, Ridge Regression and Feed-Forward Neural Networks", this article uses three different data mining methods to quantitatively trade 10 simulated time series and 10 real time series of currency exchange rates. According to the theory of stochastic processes and optimization, principal component analysis (PCA) was used to reduce the dimensionality of the prediction set, and to optimize the parameters of the verification sample, and finally the currency exchange rate strategy with the highest profit and the most stable was screened out. The 2018 article "A novel data-driven stock price trend prediction system" proposed a novel stock price trend prediction system that can predict the changes and growth rates of stock prices within a preset forecast time ( or rate of decline) interval. It uses an unsupervised heuristic algorithm to cut the raw transaction data of each stock into multiple predefined fixed-length fragments. These are employment and research opportunities that we can get after studying these two elective courses. They can not only improve our knowledge level, but also broaden our future career path and research direction.\\
    \\
    In summary, these two elective courses should be the most suitable courses for artificial intelligence based on the case of quantitative investment.



	
	
	\section{Contributions}
	
	\section{Bibliography}
	
\end{document}
